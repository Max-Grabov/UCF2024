\documentclass{article}

\usepackage[english]{babel}
\usepackage{enumitem}

\usepackage[letterpaper,top=2cm,bottom=2cm,left=3cm,right=3cm,marginparwidth=1.75cm]{geometry}

\usepackage{amsmath}
\usepackage{graphicx}
\usepackage[colorlinks=true, allcolors=blue]{hyperref}

\title{Homework Set 1}
\author{Max Grabov}

\begin{document}
\maketitle

\subsection*{Question 5}
\begin{enumerate}[label=\alph*)]
\item
First, recognize that the Centripetal force would equal the Coulomb force for the proton electron system.

\[\frac{k|q_1q_2|}{r^2} = \frac{m_ev_e^2}{r}\]
The values for $q_1$ and $q_2$ both have a magnitude of $e$, using this and some simple algebra, we can simplify the expression:

\[ke^2 = m_ev_e^2r\]
Now, using the fact that angular momentum is the cross of the radius and linear momentum, we also have the expression:

\[L=r\times{p}\]
\[=m_ev_er\]
We know that Bohr's quantization of Angular Momentum is $L = n\hbar$. This means that setting these equal to eachother:

\[m_ev_er = n\hbar\]
Going back to the original equation, we can get an expression for velocity:

\[ke^2 = m_ev_e^2r\]
\[ke^2 = v_e(m_ev_er)\]
\[ke^2 = v_en\hbar\]
\[v_e = \frac{ke^2}{n\hbar}\]
Going back to the original equation:

\[\frac{ke^2}{r^2} = \frac{m_ev_e^2}{r}\]
\[\frac{ke^2}{r^2} = \frac{m_e(\frac{ke^2}{n\hbar})^2}{r}\]
\[\frac{1}{r^2} = \frac{m_eke^2}{n^2\hbar^2r}\]
let $n_q = n^2$
\[r = \frac{n_q\hbar^2}{m_eke^2}\]
We can verify this by plugging in the values given, that it will equal the Bohr radius of a hydrogen atom in ground state

\[r = \frac{1\times{(1.055\times{10^{-34})}^2}}{9.11\times{10^{31}}\times{(1.6\times{10^{-19})}^2} \times{8.98 \times{10^{9}}}} =5.31\times{10^{-11}} \approx 5.29\times{10^{-11}}\]

\item

\end{enumerate}
\end{document}`